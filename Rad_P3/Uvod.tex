% Options for packages loaded elsewhere
\PassOptionsToPackage{unicode}{hyperref}
\PassOptionsToPackage{hyphens}{url}
%
\documentclass[
]{article}
\usepackage{amsmath,amssymb}
\usepackage{iftex}
\ifPDFTeX
  \usepackage[T1]{fontenc}
  \usepackage[utf8]{inputenc}
  \usepackage{textcomp} % provide euro and other symbols
\else % if luatex or xetex
  \usepackage{unicode-math} % this also loads fontspec
  \defaultfontfeatures{Scale=MatchLowercase}
  \defaultfontfeatures[\rmfamily]{Ligatures=TeX,Scale=1}
\fi
\usepackage{lmodern}
\ifPDFTeX\else
  % xetex/luatex font selection
\fi
% Use upquote if available, for straight quotes in verbatim environments
\IfFileExists{upquote.sty}{\usepackage{upquote}}{}
\IfFileExists{microtype.sty}{% use microtype if available
  \usepackage[]{microtype}
  \UseMicrotypeSet[protrusion]{basicmath} % disable protrusion for tt fonts
}{}
\makeatletter
\@ifundefined{KOMAClassName}{% if non-KOMA class
  \IfFileExists{parskip.sty}{%
    \usepackage{parskip}
  }{% else
    \setlength{\parindent}{0pt}
    \setlength{\parskip}{6pt plus 2pt minus 1pt}}
}{% if KOMA class
  \KOMAoptions{parskip=half}}
\makeatother
\usepackage{xcolor}
\usepackage[margin=1in]{geometry}
\usepackage{graphicx}
\makeatletter
\def\maxwidth{\ifdim\Gin@nat@width>\linewidth\linewidth\else\Gin@nat@width\fi}
\def\maxheight{\ifdim\Gin@nat@height>\textheight\textheight\else\Gin@nat@height\fi}
\makeatother
% Scale images if necessary, so that they will not overflow the page
% margins by default, and it is still possible to overwrite the defaults
% using explicit options in \includegraphics[width, height, ...]{}
\setkeys{Gin}{width=\maxwidth,height=\maxheight,keepaspectratio}
% Set default figure placement to htbp
\makeatletter
\def\fps@figure{htbp}
\makeatother
\setlength{\emergencystretch}{3em} % prevent overfull lines
\providecommand{\tightlist}{%
  \setlength{\itemsep}{0pt}\setlength{\parskip}{0pt}}
\setcounter{secnumdepth}{-\maxdimen} % remove section numbering
\usepackage{amsmath}
\usepackage{amssymb}
\usepackage{graphicx}
\ifLuaTeX
  \usepackage{selnolig}  % disable illegal ligatures
\fi
\usepackage{bookmark}
\IfFileExists{xurl.sty}{\usepackage{xurl}}{} % add URL line breaks if available
\urlstyle{same}
\hypersetup{
  pdftitle={Modeliranje evolucije populacija gujavica},
  pdfauthor={Vaše Ime},
  hidelinks,
  pdfcreator={LaTeX via pandoc}}

\title{Modeliranje evolucije populacija gujavica}
\author{Vaše Ime}
\date{2024-12-09}

\begin{document}
\maketitle

\section{Uvod}\label{uvod}

Evolucija i dinamika populacija temeljni su fenomeni u biologiji,
ekologiji i evolucijskoj teoriji. Gujavice (\emph{Oligochaeta}) igraju
ključnu ulogu u ekosustavima, posebno u procesima poput razgradnje
organske tvari, stvaranja humusa i povećanja plodnosti tla. Unatoč
njihovoj važnosti, detaljno razumijevanje dinamike populacije gujavica
pod utjecajem evolucijskih procesa, poput mutacija i prirodne selekcije,
još uvijek je ograničeno. U ovom radu, prilagođujemo stohastički model
evolucije podvrsta (Roy \& Tanemura, 2019) kako bismo istražili dinamiku
populacije gujavica i njihovu prilagodbu specifičnim uvjetima okoliša.

\subsection{Matematički okvir
modela}\label{matematiux10dki-okvir-modela}

Model se temelji na sljedećim osnovnim procesima:

\begin{enumerate}
\def\labelenumi{\arabic{enumi}.}
\tightlist
\item
  \textbf{Rođenje}: Nova jedinka može biti mutant ili može naslijediti
  fitness od postojeće populacije.
\item
  \textbf{Mutacija}: Nova jedinka dobiva nasumičnu vrijednost fitnessa u
  intervalu \([0, 1]\).
\item
  \textbf{Smrt}: Jedinke s najnižim fitnessom imaju najvišu vjerojatnost
  smrtnosti.
\end{enumerate}

\subsubsection{Formalni opis modela}\label{formalni-opis-modela}

Neka je populacija u trenutku \(n\) opisana skupom: \[
X_n = \{(k_i, f_i) : k_i \geq 1, f_i \in [0, 1], i = 1, \dots, l \},
\] gdje je \(k_i\) veličina populacije na fitness razini \(f_i\), a
\(l\) ukupan broj razina fitnessa u populaciji.

Ukupna populacija u trenutku \(n\) dana je: \[
N_n = \sum_{i=1}^l k_i.
\]

\paragraph{Proces rođenja}\label{proces-roux111enja}

S vjerojatnošću \(p\) dolazi do rođenja nove jedinke. Nova jedinka može:
1. Biti mutant s fitness parametrom \(f \sim U(0,1)\) (uniformna
distribucija). 2. Naslijediti fitness \(f_i\) od postojeće populacije s
vjerojatnošću proporcionalnom veličini populacije na toj razini
fitnessa: \[
P(f = f_i) = \frac{k_i}{N_n}, \quad i = 1, \dots, l.
\]

\paragraph{Proces smrti}\label{proces-smrti}

S vjerojatnošću \(1-p\) dolazi do smrti jedinke. Jedinka na fitness
razini s najmanjim \(f\) ima najveću vjerojatnost za smrtnost: \[
P(smrtnost|f_i) \propto \frac{1}{f_i}.
\]

Uklanja se jedna jedinka iz populacije s najmanjim fitnessom.

\paragraph{Fazni prijelaz}\label{fazni-prijelaz}

Model pokazuje fazni prijelaz definiran kritičnom vrijednošću fitnessa
\(f_c\): \[
f_c = \frac{1-p}{pr},
\] gdje: - Ako \(f < f_c\), populacija na toj razini fitnessa povremeno
izumire. - Ako \(f > f_c\), populacija raste i koncentrira se u
intervalu \([f_c, 1]\).

\subsubsection{Markovljev opis}\label{markovljev-opis}

Proces \(X_n\) je Markovljev lanac s prijelaznim vjerojatnostima: - Ako
\(X_n = \{(k, f)\}\): \[
  P(X_{n+1} = \{(k+1, f)\} | X_n) = p \cdot r \cdot U(f),
  \] gdje je \(U(f)\) uniformna distribucija. - Ako je
\(X_n = \emptyset\), populacija se ponovno pokreće nasumičnim fitnessom.

\subsubsection{Proširenje modela za
gujavice}\label{proux161irenje-modela-za-gujavice}

Za modeliranje populacija gujavica, dodaju se specifične biološke
karakteristike: 1. \textbf{Faze životnog ciklusa}: Kokoni, mlade
jedinke, odrasle jedinke. 2. \textbf{Reproduktivna dinamika}: Samo
odrasle jedinke pridonose populaciji kroz rođenje novih jedinki. 3.
\textbf{Ekološki faktori}: Utjecaj kvalitete tla i temperature na
fitness.

Dodaju se dodatni parametri kao što su: - Stopa prelaska iz faze kokona
u mladu jedinku (\gamma). - Stopa reprodukcije odraslih jedinki (\beta).

Evolucija svake faze opisana je sustavom diferencijalnih jednadžbi: \[
\begin{align*}
\frac{dC}{dt} &= -\gamma C, \\
\frac{dJ}{dt} &= \gamma C - \delta J, \\
\frac{dA}{dt} &= \beta A - \mu A,
\end{align*}
\] gdje su: - \(C\): broj kokona, - \(J\): broj mladih jedinki, - \(A\):
broj odraslih jedinki, - \(\gamma\): stopa prelaska u mlade jedinke, -
\(\delta\): smrtnost mladih jedinki, - \(\mu\): smrtnost odraslih
jedinki.

\subsubsection{Ciljevi rada}\label{ciljevi-rada}

Glavni ciljevi ovog rada su: 1. Prilagoditi opisani model za simulaciju
populacije gujavica. 2. Identificirati uvjete pod kojima populacija
ostaje stabilna ili izumire. 3. Istražiti distribuciju fitnessa u
populaciji tijekom vremena.

Simulacije će biti provedene u programskom jeziku R, a rezultati će
obuhvaćati analizu dinamike populacije, distribuciju fitnessa i
dugoročnu stabilnost populacije. Svi dijelovi koda bit će priloženi u
dodatnim materijalima.

\end{document}
